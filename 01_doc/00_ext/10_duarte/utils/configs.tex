%%--------General Document Configurations-------%%

% Essential to make glossaries (nomenclatures, acronyms)
\newglossarystyle{adjustedlong}{%
\setglossarystyle{long}% base this style on the long style
\renewcommand{\glsgroupskip}{\noalign{\penalty-50\bigskip}}% reduce in between groups spacing
}

\makeglossaries
%!TEX root = main.tex

%%-----How to use acronyms--------%%
% \newacronym{label}{short}{long}
% \acrlong - prints the description
% \acrshort - prints the acronym
% \acrfull - prints both

\newacronym{TDC}{TDC}{time-to-digital converter}
\newacronym{TDL}{TDL}{tapped delay line}
\newacronym{WU}{WU}{wave-union}
\newacronym{RO}{RO}{ring oscillator}
\newacronym{DLL}{DLL}{delay-locked loop}
\newacronym{ADDLL}{ADDLL}{all-digital delay-locked loop}

\newacronym{FPC}{FPC}{four-phase clock}
\newacronym{DS}{DS}{dual sample}

\newacronym{VCDL}{VCDL}{voltage-controlled delay line}
\newacronym{DCDL}{DCDL}{digital-controlled delay line}

\newacronym{FPGA}{FPGA}{field programmable gate array}
\newacronym{ASIC}{ASIC}{application-specific integrated circuit}
\newacronym{CMOS}{CMOS}{complementary metal–oxide–semiconductor}

\newacronym{DNL}{DNL}{differential nonlinearity}
\newacronym{INL}{INL}{integral nonlinearity}
\newacronym{PVT}{PVT}{process, voltage and temperature}

\newacronym{LSB}{LSB}{least significant bit}
\newacronym{RMS}{RMS}{root mean square}
\newacronym{STA}{STA}{static time analysis}


\newacronym{PLL}{PLL}{phase-locked loop}
\newacronym{FLIM}{FLIM}{fluorescence lifetime imaging microscopy}
\newacronym{LiDAR}{LiDAR}{light detection and ranging}
\newacronym{ToF}{ToF}{time-of-flight}

\newacronym{FF}{FF}{flip-flop}
\newacronym{LUT}{LUT}{lookup table}
\newacronym{CLB}{CLB}{configurable logic block}
\newacronym{ADC}{ADC}{analog-to-digital converter}
\newacronym{DAC}{DAC}{digital-to-analog converter}

\newacronym[firstplural=read access memories (RAMs)]{RAM}{RAM}{read access memory}

\newacronym{RBC}{RBC}{reflected binary code}




% Font selection
\setmainfont[Mapping=tex-text, AutoFakeSlant=0.35,
Path = utils/Fonts/,
UprightFont = *-regular,
BoldFont = *-bold,
BoldItalicFont= *-bold,
BoldItalicFeatures={FakeSlant=0.35}
]{NewsGotT}

% Line spacing
\renewcommand{\baselinestretch}{1.5}

% Page formatting
\pagestyle{fancy} %check the fancyhdr package documentation for more option
\fancyhf{}
%\fancyhead[LO,RE]{\nouppercase{\leftmark}}
%\fancyhead[RO,LE]{\thepage}
%\fancyhead[LE,RO]{\rmfamily\thepage}%
\fancyhead[RE]{\slshape\nouppercase{\rightmark}}
\fancyhead[LO]{\slshape\nouppercase{\leftmark}}%
\fancyfoot[C]{\thepage}
\renewcommand{\headrulewidth}{1pt}
\renewcommand{\footrulewidth}{0pt}
\setlength{\headheight}{15pt}

% Chapter spacing -- comment for default chapter spacing
\titleformat{\chapter}[hang]
    {\normalfont\huge\bfseries}{\chaptertitlename\ \thechapter:}{1em}{}
\titlespacing*{\chapter}{0pt}{0pt}{20pt} %left - upper -- lower

% Bibliography as References
\renewcommand\bibname{References}

% Bad hyphenation corrections -- correct hyphenations that may appear wrong throughout the text
\hyphenation{op-tical net-works semi-conduc-tor mo-du-les pro-jects }

% non-SI units needed
\sisetup{inter-unit-product = {}, detect-weight=true, allow-number-unit-breaks=true}
\DeclareSIUnit{\dBm}{dBm}
\DeclareSIUnit{\eur}{\text{\euro}}
\DeclareSIUnit{\Vp}{V\textsubscript{p-p}}
\DeclareSIUnit{\S}{S}
\DeclareSIUnit{\day}{day}
\DeclareSIUnit{\bps}{\text{b}\text{p}\second}

% Code aspect configurations

%%% Example configurations, change at your own taste
\renewcommand{\lstlistingname}{Code}% Listing -> Code
\renewcommand{\lstlistlistingname}{\lstlistingname \ Snippets} % Code Snippets on ToC

\definecolor{stringcolor}{RGB}{153,0,153}
\definecolor{commentcolor}{RGB}{76,153,0}
\definecolor{codebackgroundcolor}{RGB}{242,242,242}

\lstset{
    backgroundcolor=\color{codebackgroundcolor},   % choose the background color; you must add \usepackage{color} or \usepackage{xcolor}; should come as last argument
    basicstyle=\footnotesize\ttfamily,        % the size of the fonts that are used for the code
    %basicstyle=\ttfamily,
    breakatwhitespace=false,         % sets if automatic breaks should only happen at whitespace
    breaklines=true,                 % sets automatic line breaking
    captionpos=b,                    % sets the caption-position to bottom
    commentstyle=\color{commentcolor}\footnotesize,
    %morecomment=[l][\color{dandelion}]{\#},   % comment style
    deletekeywords={...},            % if you want to delete keywords from the given language
    extendedchars=true,              % lets you use non-ASCII characters; for 8-bits encodings only, does not work with UTF-8
    frame=leftline,	                   % adds a frame around the code
    keepspaces=true,                 % keeps spaces in text, useful for keeping indentation of code (possibly needs columns=flexible)
    keywordstyle=\color{blue},       % keyword style
    language=C,                 % the language of the code
    directivestyle={\color{brown}},
    morekeywords={*,...},            % if you want to add more keywords to the set
    numbers=left,                    % where to put the line-numbers; possible values are (none, left, right)
    numbersep=5pt,                   % how far the line-numbers are from the code
    %numberstyle=\tiny\color{mygray}, % the style that is used for the line-numbers
    rulecolor=\color{black},         % if not set, the frame-color may be changed on line-breaks within not-black text (e.g. comments (green here))
    showspaces=false,                % show spaces everywhere adding particular underscores; it overrides 'showstringspaces'
    showstringspaces=false,          % underline spaces within strings only
    showtabs=false,                  % show tabs within strings adding particular underscores
    stepnumber=1,                    % the step between two line-numbers. If it's 1, each line will be numbered
    stringstyle=\color{stringcolor},     % string literal style
    tabsize=2	                   % sets default tabsize to 2 spaces
    %title=\lstname                   % show the filename of files included with \lstinputlisting; also try caption instead of title
}

%%---------Thesis Information--------------------%%

\newcommand{\myparagraph}[1]{
    \paragraph{#1}\mbox{} \bigskip
}

% \newcommand{\myparagraph}[1]{
%     \parbox{50pt}{#1}
% }