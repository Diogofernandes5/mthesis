%!TEX root = ../main.tex

The goal of improving the \gls{PVT} calibration mechanisms for \glspl{TDC} is to maintain high performance levels in the face of variations in \gls{PVT} conditions. As \glspl{TDC} are integrated into more applications, the impact of \gls{PVT} variations on \gls{TDC} linearity will become more significant. Therefore, it is important to develop effective calibration mechanisms and methods that can reduce these variations and maintain the high performance levels required by modern applications.

Recent advances in \gls{TDC} technology have made it possible to achieve high performance levels in a variety of applications. The rapid pace of technological change and the need for quick time-to-market had led to a focus on reconfigurable and customizable \gls{TDC} architectures. The integration of microprocessors on most \gls{FPGA} also presents an opportunity for automatically generated \glspl{TDC}. An algorithm can be implemented on the microprocessor to analyze the non-linearity of the \gls{TDC}, and the results can be used to automatically rearrange the hardware to improve its linearity and reduce missing codes. Furthermore, the use of machine learning techniques could potentially enable \glspl{TDC} to automatically optimize their performance based on real-time data.

In addition to these advances, higher sampling rate architectures are required to meet the demands of modern applications. \gls{TDC} architectures based on \glspl{TDL} typically have sampling rates equal to the frequency of the reference clock being used. An alternative to reduce other architectures dead time, such as pulse shrinking and \acrlongpl{RO}, is to have multiple \gls{TDC} channels operating in an interleaved schema. However, this solution can have a negative impact on resource utilization. Therefore, phased clocks and \glspl{TDL} are more appropriate for applications with high input event rates.

Furthermore, as technology continues to scale down and \gls{FPGA} platforms improve, it is expected that \gls{TDL} architectures will achieve higher resolutions due to reductions in the cells propagation delays. However, this will also exacerbate the negative effects of \gls{PVT} variations on the linearity of \glspl{TDL}. A shorter delay line ensures better linearity and is less susceptible to changes of supply voltage and temperature, so hybrid \glspl{TDL} and differential delay lines architectures can help to mitigate these problems, although they increase implementation complexity. Counter based \gls{TDC} architectures offer great linearity. Even phased clock architectures do not significantly suffer much from \gls{PVT} variations since the use of an internal \gls{PLL} provides automatic stabilization against changes of ambient temperature and supply voltage. However, while these architectures have great linearity against \gls{PVT} variations, the maximum achievable resolution is their main drawback, as it may not met the requirements of many applications.

In conclusion, the evolution of \gls{TDC} technology will continue to be driven by the requirements of applications. \gls{FPGA}-based \glspl{TDC} have recently demonstrated performance levels that can compete with those of \gls{ASIC}-based \glspl{TDC}, and it is expected that they will become more popular and start to be used in commercial products. This will increase the popularity, usability of \glspl{TDC} and reduce production costs.
